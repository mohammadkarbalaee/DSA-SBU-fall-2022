\documentclass[twocolumn,a4paper, 10pt]{article}
\usepackage{graphicx}
\usepackage{fullpage}
\usepackage{titling}

\title{\textbf{Data Structures and Algorithms} \\ \emph{1st Session}}
\author{\underline{Lecturer: Dr.Katanforoush} \\ \small{Author: Muhammad Karbalaee}}

\begin{document}
	\pretitle{%
		\begin{center}
		\LARGE
		\includegraphics[scale=0.3]{../../assets/images/logo.jpg} \\
	}
	\posttitle{\end{center}}
	\maketitle
    \tableofcontents

    \clearpage
    \section{Different sorts of problem statement in algorithm theory}
        \subsection{What is an algorithm?}
            Algorithm is the method of solving an specific problem.
        \subsection{Importance of problem categorization}
            The definition of algorithm is mostly relied on what a problem is.
            Therefore, knowing the exact definition of problem is vital to understand
            what an algorithm is.
            In theory of algorithms,only specific statements are valid and every problem
            should get pigeonholed in on of the categories. There are four major problem sorts.
            \subsubsection{Satisfaction Problems}
                Problems in this category are stated in the following pattern. 


                \emph{P(X) is a proposition in the predicate logic \footnote{Predicate logic in simple words means a logic system that
                contains variables in its statements and is concerned with the existence of that variable not the value of the variable itself 
                for example: ather than propositions such as "Socrates is a man", one can have expressions in the form "there exists x such that x is Socrates and x is a man", where "there exists" 
                is a quantifier, while x is a variable.} and the question is wether there is an X that P(X)=True. } 


                The algorithm which solves a satisfaction problem can only have two outputs, True or False.
                It's crucial to know that in satisfaction problems, the algorithm is not concerned with 
                finding the wanted X, but is focused on validating the existence of X. \\ \\ 
                \underline{Example 1}
                    \emph{Is 2 a prime number?} \\ \\ 
                        In this example P is being prime number and X=2 \\ 
                        This problem could be stated in other words too, which is still a satisfaction problem. \\ 
                        \emph{Is P(X)=True by the given value X?} \\ \\
                \underline{Example 2}
                    \emph{Consider graph G with V as its vertex set and E as its edge set. Is there a path between s and t where \begin{math}
                    s,t \in V \end{math}}
            \subsubsection{Root finding Problems}
                    Remember in the previous category the emphasis was on on the existence of X? \\ 
                    In contrast with that, root finding problems are concerned with finding the X itself not checking wether it exists or not. \\ 
                    Formally stating: 

                    
                    \emph{Find an X, if exists, such that P(X)=True} \\ \\ 
                    \underline{Example }
                        \emph{Consider graph G with V as its vertex set and E as its edge set. Find path between s and t such as 
                        \begin{math}
                        P=<v_1, v_2, v_3, ...., v_k>, k\in N \end{math} where \begin{math}
                        s,t \in V \end{math}}
            \subsubsection{Enumeration Problems}
                    These problems aim to find all Xs that P(X)=True, not just one. Each enumeration problem can be 
                    reduced to a root finding problem and root finding problems can be reduced to 
                    a satisfaction problem if the working space is countable. \footnote{Problem reduction means, by having an algorithm that solves problem A,
                    it's possible to come up with an algorithm that solves problem B, this way it is said that problem B is reduced to problem A.
                    For example: To find all two-digit prime numbers, we should be able to firstly, find a prime number. So the first problem is 
                    reduced to the second one.} \\ \\
                    \underline{Example 1 }
                        \emph{Find all two-digit prime numbers}
            \subsubsection{Optimization Problems}
                    Consider f as a function defined this way,
                    \begin{math}
                         f:S \Rightarrow R
                    \end{math} and P(X) as a satisfiable proposition. \\ 
                    Optimization problems are formed this way then: \\ 
                    \emph{Find X such that f(X) reaches its minimum/maximum value and P(X)=True} \\ \\
                    f: Object function \\
                    P: constraints \\ \\

                    \underline{Example }
                        G(V,E) is a graph and w is an edge weighing function such as \begin{math}
                            w: E\Rightarrow R
                        \end{math}. \\ 
                        For given s and t such that \begin{math}
                            s,t \in V
                        \end{math}, find the path with minimum total weight from s to t. \\ \\ 
                        S: The space of all paths in G(V,E) \\
                        \begin{math}
                            X \in S, Path \in S \\ 
                            f(x) = \sum_{i = 1}^{k} w(v_i,v_{i+1}) \\ 
                            (v_i,v_{i+1}) \in Path \\ 
                            P: v_1 = s,     v_k = t
                        \end{math}

            \clearpage

    \section{Exercises}
        Here are a couple exercises related to the subject of the first session. \\ \\ 
        \emph{Categorize each problem into one of the four categories you learnt in this session.} \\ \\
        1.Consider M as 3 by 5 matrix. Is M invertible? \\
        2.Find an X such that \begin{math}
            2x + 10 = 0
        \end{math} \\ 
        3.Find all divisors of 35. \\ 
        4.All prime numbers are odd. \\ 
        5.What is the fastest way to get from the dorm to the campus. \\
        6.Consider V as a vector two-dimensional space. Find all vectors in V such as t that \begin{math}
            t = <0,0>
        \end{math} \\ 
        \clearpage
    \section{Solutions}
        1.satisfaction problem \\ 
        2.root finding problem \\ 
        3.enumeration problem \\ 
        4.satisfaction problem \\ 
        5.Optimization problem \\ 
        6.enumeration problem \\
\end{document}